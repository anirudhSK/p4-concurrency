\section{Frequently asked questions}
\label{s:faq}

\begin{enumerate}
\item What happens when the controller changes a table entry while a packet is
being processed in the data plane? \\
This draft doesn't deal with interactions between the control and data plane,
because the control plane isn't in P4 and P4 cannot easily mandate its
behavior. A reasonable semantics for this case is to say that a packet either
sees the new table entry or the old one, but not a muddled combination of the
two.

\item What about nested atomic blocks?\\
We can take one of two equivalent approaches. Either, we forbid nesting atomic
blocks within each other, or the compiler removes all nested atomic blocks and
preserves only the outermost atomic annotation. This is the SQL approach to
nested transactions (the database analogue of nested atomic blocks).  Here one
transaction calls a stored procedure, which might itself run as a transaction.
When the stored procedure (the inner transaction) commits, it has no effect
until the outer transaction commits, at which point it is commited as well
(\url{https://technet.microsoft.com/en-us/library/ms189336(v=sql.105).aspx}).

\item Atomic guarantees sound hard to support?\\
Yes, they probably are. They require work in the compiler to support the
general case of user-defined atomic blocks, which go beyond the default atomic
annotations. They also require work in the compiler to provide reasonable error
messages when an atomic block is rejected.

However, I contend it is better than the alternative, which is to expect the
programmer to invoke target-specific atomic methods. This is both unportable
and requires a target implementer to unnecessarily reveal the target's details
to the programmer. Besides, implementing the required logic in the compiler
isn't all that hard: it is around a few hundred lines of code to do the
required program analysis to whittle down a programmer's atomic block to
minimal atomic blocks. To deal with the problem of rejecting large atomic blocks,
we could conservatively restict the complexity of
the code within the atomic block, just like we do with the complexity of the
expression within P4's \texttt{verify} statement today.

\item Are we going to have an entire spectrum of weak/relaxed atomicity models,
like every language that has gone down the garden path of atomic blocks?\\
No. In cases where this has happened (such as C++:
\url{https://3f993110-a-62cb3a1a-s-sites.googlegroups.com/site/tmforcplusplus/C\%2B\%2BTransactionalConstructs-1.1.pdf}
and \url{http://en.cppreference.com/w/cpp/language/transactional\_memory}),
this was the result of interoperability with existing forms of synchronization
such as locks, volatiles, condition variables, and mutexes, which didn't
provide the same guarantees as atomic blocks, and which could be accessed both
within atomic and non-atomic blocks.

This spectrum of models is not fundamental. For instance, Haskell's STM
implementation (\url{http://community.haskell.org/~simonmar/papers/stm.pdf})
doesn't have relaxed or weaker forms of atomicity. We can take the same
approach with P4 and make atomic the only form of concurrency.

\item What about exceptions within @atomic block? \\
Typically, database transactions abort on exceptions.  It's as though the
transaction never ran. P4 doesn't have an exception handling mechanism (the
verify statement within the parser state machine can be viewed as syntactic
sugar for transitioning to the reject state, just like any other statement that
transitions to the reject state in the parser state machine). Without exception
handling, every @atomic block runs to completion.

\item Can operations on local externs be made atomic by default? \\
Yes. Except there is no such language construct for local externs today. Local
variables within a parser body or a control block's apply method cannot be
externs. All externs declared within a parser or control declaration are
global and shared by all instances by default. This would be useful for externs
such as parser checksums that run an independent instance for each thread.

% 5. Unstructed control flow: returns, exits
\end{enumerate}
