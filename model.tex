\section{Model}

We first specify a formal model for concurrency below. We discuss it at an
abstract level, later show how it can applied to various productions within the
P4 grammar, and finally discuss compilation strategies.

An instance of a parser or control block type (\S9.2.8 of the spec) is called a
parser or control block instance.  All computations within P4 happen inside
these instances as specified by the statements within them (\S12).  We say a
parser or control block is ``invoked'' when some external event triggers the
execution of a parser instance or a control block instance.  This external
event could be the arrival of an unparsed packet (\texttt{packet\_in}) for a
parser instance or parsed headers (\texttt{in} or \texttt{inout} function
arguments) for a control block.

When an external event invokes a parser or control block instance, we
semantically think of the instance launching a new thread to handle that event
in the background, leaving the instance free to handle the next event. The
concurrency model doesn't regulate the implementation. For instance, the number
of threads in a thread pool is left to the implementation.  In fact, the
implementation doesn't even need to use a thread pool. It can instead rely on
pipelined processing, where a small portion of the work for each event is
handled by one thread, which then passes it off to the next thread (e.g.,
match-action pipelines).

Each of the launched threads runs to completion, executing all statements
within the control block or parser instance one after another, sequentially.
Once the thread completes processing an event, it triggers another parser or
control block, as specified by the target architecture.

Any interleaving of threads is permitted as long as statements within a thread
aren't reordered. The atomic units of execution within each thread's statements
are defined, in principle, by providing an \texttt{@atomic} annotation around
specific statements or expressions in the program. This annotation denotes that
the computations within those statements or expressions appear to execute
atomically/instantaneously.

In practice, the P4 compiler annotates many expressions and statements atomic
 by default, to reduce the burden on the P4 programmer.
Statements and expressions that are atomic by default are all statements and
expressions that do not access an extern instance, i.e., statements or
expressions that read or write the following types of data:
\begin{enumerate}
\item intrinsic metadata (\S4)
\item metadata (\S4)
\item packet headers (\S4)
\item local variables declared within a control block's apply block, a parser's
state, or an action's body (\S11.2)
\end{enumerate}
We also include statements that recursively only include statements that read
or write the above types of data, such as a statement that calls an action,
which respects the conditions above.

The rationale for not including extern instances is that extern instances
encapsulate data-plane state, unlike the scenarios above that don't include any
state visible to all threads. State modification through extern method calls
cannot be declared atomic by default because depending on the extern method,
some intermediate state may be visible to another thread, which contradicts the
requirement that modifications be instantaneous. We recommend that method calls
on extern instances be implemented as atomic by target architecture
implementers, e.g., serializing all calls to a counter or register to provide
the illusion of atomicity.

Certain combination of statements are also annotated atomic by default. As an
example, consider the statements:
\begin{verbatim}
x = x & 4w0;
\end{verbatim}

and
\begin{verbatim}
x = 4w0 & x;
\end{verbatim},
where x is a local variable. Then the block statement:

\begin{verbatim}
{
  x = x & 4w0;
  x = 4w0 & x;
}
\end{verbatim}
is atomic by default because it calls no extern instance.

Certain categories of statements are not atomic by default. In particular, it
is recommended that individual method calls on extern instances be atomic, a
pair of calls to two different extern instances is not atomic by default. An
example is a pair of counters, say, \texttt{c1} and \texttt{c2}. While the
increment method call for each counter appears to execute atomically, other
statements can be interleaved between the increments to \texttt{c1} and
\texttt{c2}, unless the programmer encloses both method calls within one atomic
annotation.

The above discussion should give a sense for how atomic is used in practice. To
reduce programmer burden, we recommend that P4 compilers implicity or
explicitly implement a inference procedure for conservatively inferring whether
any given production within a grammar is atomic by default. The procedure could
be modeled based on some of the rules given above.
%TODO: We would ideally provide a full inference procedure in the appendix.
