\section{Motivation}
P4 lacks a formal concurrency model. This is important when considering
interactions between multiple packet processors in the data plane. These packet
processors could be match-action tables, pipelines, or cores. Further, these
packet processors can share state. As an example, consider flowlet switching
from the SIGCOMM 2015 P4 tutorial. The state stored in register
\texttt{flowlet\_id} is shared by two tables:

\begin{enumerate}

\item It is read in the action \texttt{lookup\_flowlet\_map} in the
\texttt{flowlet} table

\item It is later written in the action \texttt{update\_flowlet\_id} in the
\texttt{new\_flowlet} table.

\end{enumerate}

Here's the relevant code snippet for flowlet switching, written in P4-16.
\begin{verbatim}
control ingress {
  ...
  ...
  apply {
    apply(flowlet);
    if (ingress_metadata.flow_ipg > FLOWLET_INACTIVE_TOUT) {
        apply(new_flowlet);
    }
    apply(ecmp_group);
    apply(ecmp_nhop);
    apply(forward);
  }
}
\end{verbatim}
In the snippet above, \texttt{flowlet} reads a piece of state
(\texttt{flowlet\_id}) that \texttt{new\_flowlet} writes. Because there is no
support for locking within P4, this could lead to a race, where a second packet
reads an old value of \texttt{flowlet\_id}. It is unclear what behavior to
expect for these programs because P4 lacks a formal concurrency model when
state is modified in the data plane. This draft is an attempt at fixing this.
