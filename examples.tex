\section{Examples}

With the atomic annotation, the flowlet switching example can now be written in
one of two ways. We later discuss pros and cons of each approach.

\subsection{Using registers}

\begin{verbatim}
control ingress {
  ...
  ...
  apply {
    @atomic{
      apply(flowlet);
      if (ingress_metadata.flow_ipg > FLOWLET_INACTIVE_TOUT) {
          apply(new_flowlet);
      }
    }
    apply(ecmp_group);
    apply(ecmp_nhop);
    apply(forward);
  }
}
\end{verbatim}

This has the correct behavior of \textit{atomically} reading the register
\texttt{flowlet\_id}, and updating it (by incrementing it:
\url{https://github.com/p4lang/p4c/blob/master/testdata/p4_14_samples_outputs/flowlet_switching.p4#L145})
if the condition within the if clause is true. It assumes the standard library
extern element \texttt{register} (\S8.5.2) supports an atomic read and write,
but nothing more.

Underneath, a compiler would have to turn this conditional read-modify-write
into a hardware instruction that atomically updates a piece of memory if a
condition is true. The compiler section of this draft discusses how this
compiler would work.

\subsection{Using more complex extern types}

We could imagine an extern type that provides a thin wrapper around a
hardware instruction that allows a program to atomically increment a state
variable if a boolean condition is true.  A possible declaration for this
extern type is given below.

\begin{verbatim}
extern Conditional++ {
  @atomic void conditional_inc(bool condition);
  @atomic int read();
}
\end{verbatim}

Here, \texttt{conditional\_inc} is declared an atomic method call by the
implementer of the target architecture who specifies the behavior of
\texttt{Conditional++}.  The behavior of \texttt{Conditional++}'s
\texttt{conditional\_inc} method is to atomically increment an internal state
variable if {\texttt condition} is true.

With \texttt{Conditional++}, flowlet switching can now be written as:
\begin{verbatim}
control ingress {
  // Declare a Conditional++ object shared across all control block threads
  Conditional++ conditional++;
  apply {
    conditional++.conditional_inc(ingress_metadata.flow_ipg > FLOWLET_INACTIVE_TOUT);
    apply(ecmp_group);
    apply(ecmp_nhop);
    apply(forward);
  }
}
\end{verbatim}

\subsection{More examples}

We list out a few more examples in both forms (using registers and using more
complex extern types). These examples are drawn from
\url{https://github.com/packet-transactions/domino-examples/tree/master/domino_programs},
which is a set of examples written in a high-level packet-processing language
called Domino. Domino is centered around user-defined atomic blocks, so using
our new @atomic notation, every Domino program can be written as an equivalent
P4 program using an enclosing @atomic.

\paragraph{Rate-Control Protocol (RCP)}

With registers alone, RCP can be written like this.
\begin{verbatim}
control ingress {
  Register<bit<32>>(1) input_traffic_Bytes;
  Register<bit<32>>(1) sum_rtt_Tr;
  Register<bit<32>>(1) num_pkts_with_rtt;
  ...
  apply {
    @atomic{
      bit<32> input_traffic_Bytes_tmp;
      input_traffic_Bytes.read(0, input_traffic_Bytes_tmp);
      input_traffic_Bytes_tmp = input_traffic_Bytes_tmp + metadata.pkt_len;
      input_traffic_Bytes.write(input_traffic_Bytes_tmp, 0);
      if (pkt_hdr.rtt < MAX_ALLOWABLE_RTT) {
        bit<32> sum_rtt_Tr_tmp;
        sum_rtt_Tr.read(0, sum_rtt_Tr_tmp);
        sum_rtt_Tr_tmp = sum_rtt_Tr_tmp + pkt_hdr.rtt;
        sum_rtt_Tr.write(sum_rtt_Tr_tmp, 0);

        bit<32> num_pkts_with_rtt_tmp;
        num_pkts_with_rtt.read(0, num_pkts_with_rtt_tmp);
        num_pkts_with_rtt_tmp = num_pkts_with_rtt_tmp + 1;
        num_pkts_with_rtt.write(num_pkts_with_rtt_tmp, 0);
      }
    }
  }
}
\end{verbatim}

With more complex extern types, it would be written as:
\begin{verbatim}
extern ConditionalAccumulator<T> {
  ConditionalAccumulator();
  void read(out T result);
  void write(in T accumuland, bool condition);
}
...
control ingress {
  counter<bit<32>>(1, CounterType::packets_and_bytes) input_traffic_Bytes;
  ConditionalAccumulator<bit<32>>(1) sum_rtt_Tr;
  ConditionalAccumulator<bit<32>>(1) num_pkts_with_rtt;
  ...
  apply {
    ... 
    @atomic{
      input_traffic_Bytes.count();
      sum_rtt_Tr(pkt_hdr.rtt, pkt_hdr.rtt < MAX_ALLOWABLE_RTT);
      num_pkts_with_rtt(1,    pkt_hdr.rtt < MAX_ALLOWABLE_RTT);
    }
  }
}
\end{verbatim}

\subsection{Comparing the two approaches}
The first approach using registers leads to portable code, because it only
assumes simple registers with atomic read and write
capabilities.\footnote{While the examples above suggest that registers lead to
overly bloated code, this could be easily fixed by providing extern methods for
registers that allows us to carry out arithmetic on registers.} It also allows
the target implementer to hide potentially propreitary information regarding
how a conditional update is actually implemented, i.e., the target implementer
doesn't need to provide a method that reveals the fact that it does a
conditional increment or a conditional accumulate.

With the second approach, the target implementer needs to specify an external
method for every stateful update that can be carried out using the target's
hardware capabilities. This could prove cumbersome.

 However, the first approach puts the burden on the compiler to figure out
exactly how to translate a block of programmer statements within a
\texttt{@atomic} into hardware instructions. It also leads to the possibility
that the programmer writes code that is rejected because the atomic block is
too large to implement using an atomic hardware instruction and guarantee
atomic semantics. This could be potentially solved by better compiler error
messages telling the programmer how to fix the program, but it adds to compiler
complexity once again.
